\documentclass[10pt,a4paper]{article}
\usepackage[utf8]{inputenc}
\usepackage[english]{babel}
\usepackage{amsmath}
\usepackage{amsfonts}
\usepackage{amssymb}
\usepackage{graphicx}
\usepackage{lmodern}
\usepackage[left=2cm,right=2cm,top=2cm,bottom=2cm]{geometry}
\author{Christophe Sauty}
\parindent=0mm
\title{Accrétion et Jets}
\begin{document}
\maketitle
\tableofcontents
\newpage
$M_{\odot}$
\section{The necessity of the Accretion-Jet Mechanism}
\subsection{Active Galactic Nuclei}
\subsubsection{Radio quiet, Seyfert galaxies}
The first spectrum of a Seyfert galaxy was taken in 1943.\\
\\
In the catalogue of 3C sources, it was first observed that there was a large population of highly blue "stars" with unknown large lines (8000 km/s), like in 3C48 (Matthews et al., 1961, AAS, Dec1960). Moreover it was noticed that those objects where redshifted. Greenstein and Schmidt, in 1964, for 3C48 and 3C273, measured redshifts of $z=0.45$ and $0.16$. Thus those objects have to be at cosmological distances but with very high luminosity.\\
\\
Seyfert galaxies are usually spiral galaxies. They have a radio core, which fluctuates in brightness. Because of the intense radio core they are classified in the larger family of Active Galactic Nuclei. A ling debate occurred initially to know if the radio emission could be constructed form star burst emission. Nowadays, the radio core and the whole spectrum (because we have multi-wavelength observations) is understood as the result of the emission of an accretion disk with or without outflows around a central supermassive black hole.\\
\\
As radio-quiet objects represent $\sim 90 \% $ of the 'average' AGN, the also represent a very large source for the gamma ray background emission at low level.\\
\\
Illustrations : NGC 4258, Seyfert galaxy in Ursa major. Supermassive black hole of 39 million Sun mass. Warped disk, two light years in diameter.\\
\\
H2O masers amplify the microwave radio emission disk edge-on orbits 2 million miles per hour.

\subsubsection{Radio Loud, Narrow Line Radio Galaxies}
Conversely to Seyfert galaxies, radio loud galaxies have an extended radio emission in addition to the radio core. They are usually seen in giant elliptical galaxies and not spiral ones. From our observational point of view, the most powerful such sources are of course quasars and BL Lac objects. However the most striking structure are seen in radio galaxies of Fanaroff Riley types.\\
\\
Colour composite image of Centaurus A, revealing the lobes and jets emanating from the active galaxy's central black hole. This is a composite of images obtained with three instruments, operating at very different wavelengths.[...]\\
\\
Radio loud sources and quasars have relativistic jets in addition to the accretion disk, with bulk Lorentz factor $\Gamma=\dfrac{1}{\sqrt{1-V^2/c^2}}$ of 3 to 10 or possibly higher.\\
\\
They are two types of radio loud galaxies, FRI and FRII.
\begin{itemize}
\item FRI radio galaxies show elongated jets on both sides, on the kpc scale terminated with two radio lobes (spatially extended structure). On the pc scale, we see one-sides jets with relativistic speeds. On the kpc, the jet is sub-relativistic and has decelerated.
\item FRII radio galaxies are one sided and relativistic on the pc and the kpc scales. The are more powerful and terminated with two hot spots. This implies that, although the visible jet is one sided, counter jet exist but is invisible because it is not relativistically beamed.
\item HYMORS is a new class of FR galaxies, witch are FRI on one side and FRII on the other side. This may be explained of the environment plays an important role.
\end{itemize}
\subsubsection{Quasars, BL Lac, QSO}
The other radio sources belong to the wider class of quasars (quasi-stellar objects). They are radio quiet as well as radio loud quasars (QSOs). As for radio galaxies, proper motions could be measured in some quasars with jets. In all cases high relativistic velocities are deduced. Quasars, BL Lac and QSO appear to be pointing towards us, which explains - because of the Doppler effect - that they are more luminous than radio galaxies.

\subsection{Young Stellar Objects}
\subsubsection{Definition}
A \textbf{pre-main sequence star} (PMS star or PMS object) is a star that as not yet reached the main sequence. In the following we shall consider mainly loss mass stars with mass less or equal to two solar masses. High mass pre-main sequence star exhibits strong outflows and winds but not well structure jets at least at the present stage of observations. Loss mass stars go through the stage of T Tauri star or FU Orionis star ($<2$ solar mass). High mass stars are usually Herbig Ae/Be stars (2-8 solar mass).
\subsubsection{HH objects and CO Outflows}
Jets in YSOs where first seen through HH objects. They were defined as bright knots initially. Usually aligned along a given axis. With the development of new instruments (in particular the HST), they appear as complex structure and often internal shocks or bow shocks. Bow shocks are characterized by the C-Shock front as seen on the extremity of HH34 below (right).\\
\\
In HH34 (and many other HH objects), the evolution seen from various successive observations allows to measure proper motion of the knots. Typical values of a few hundreds of km/s are usually inferred. However some knots may be standing shocks ans the proper motion of the shocks is not directly the speed of the underlying jets though similar velocities are measured by other means.\\
\\
These premain stars have accretion-ejection systems with typical accretion rates of $\dot{M}_{accr}=10^{-7}M_{\odot} /yr$ to $10^{-6}M_{\odot} /yr$. Around $30\%$ of these objects also have jets with typical wind mass loss rates of $\dot{M}_{jet}=10^{-8}M_{\odot} /yr$ to $10^{-7}M_{\odot} /yr$, densities $n_{jet}=10^3$ to $10^7$ cm$^{-3}$ (this is strongly model dependent) and temperatures around $10^3$ to $10^4$ K. Of course the density in the accretion disk is much higher and the temperature much lower.
\subsubsection{Molecular flows $H_2$, $CO$}
Next picture of the HST is called "Birth and death of the Milky Way". It shows a multi-wavelength picture with dark clouds illuminated and ionized by stars. Clouds are clumps resulting from the Parker instability in the Milky Way. They are themselves unstable and because of gravitational collapse they contract forming accretion torii and disks.\\
\\
Some disks (seen in infra red, red contours in next figure) show very powerful molecular outflows sometimes associated with the previous HH objects and Optical jets but not always. Molecular outflows with mass loss rates of $\dot{M}_{jet}=10^{-5}M_{\odot} /yr$, velocity a few 10 km/s and cold temperatures of a tens of Kelvin. The underlying accretion disks have accretion rates higher than typically $\dot{M}_{acrr}=10^{-4}M_{\odot} /yr$.\\
\\
HH211: Another nice example of a powerful molecular outflow with its inner accretion disk. Note that the inner structure is not very different from some pictures of Planetary Nebulae.\\
\\
Molecular outflows are collimated in the observational perspective but not as well as optical jets. From a theoretical point of view the opening angle is so large that they may be considered as uncollimated winds. The opening angle is the angle the outflow makes from its central source. They are usually surrounding the optical jet when there is one.

\subsubsection{Accretion in YSOs}
The accretion process is always present even if outflows are not. This means either that the ejection is episodic or (more likely) that accretion does not necessarily require all the time the formation of a collimated source.\\
\\
Before having disk images, the presence of an accretion disk was deduced from the infrared excess measured by the Spectral Index.
\[\text{Spectral Index:}\quad \alpha=\frac{d\log(\nu F)}{d \log(\nu)}\quad\text{where}\quad F_{\nu}\quad\text{is the flux density.}\]
For disks in YSO, the spectral index is calculated in the wavelength interval of $2.2-10 \mu$m (near infrared region).\\
\\
In 1986 Lada C.J. and Wilking B.A. were the first to established the existence of various classes now known as Class I,II,III. In 1993 Andre et al. discovered a younger class called class 0, with a strong sub-millimetre emission, but very faint at $\lambda<10\mu$m. In summary the classes are:
\begin{itemize}
\item \textbf{Class 0} sources: undetectable at $\lambda<10\mu$m
\item \textbf{Class I} sources: $\alpha>0.3$
\item \textbf{Flat spectrum} sources: $0.3>\alpha>-0.3$
\item \textbf{Class II} sources: $-0.3>\alpha>-1.6$
\item \textbf{Class III} sources: $\alpha<-1.6$
\end{itemize}
The following figure gives a schematic view of the spectrum and the role of the accretion disk (from André, 1996).\\
\\
More information on accretion disks is expected from next generation of instruments like ALMA and Herschell (HIFI).
\subsection{Need for Accretion, Eddington Luminosity}
\subsubsection{Luminosity: Only gravity can do it}
\subsubsection{Eddington Luminosity}
\subsubsection{The disk spectrum}
\subsubsection{AGN Luminosity and Spectra}

\subsection{Need for Jets}
\subsubsection{Energy in the extended lobes}
\subsubsection{Synchrotron emission}
\subsubsection{Proper motion}

\subsection{Rotation and the Angular Momentum transfer}
\subsubsection{The Angular Momentum problems in YSO}
\subsubsection{Rotation of AGN and Microquasar Central Black Hole}
\subsubsection{Conclusion, The Global Picture of the Jet}

In conclusion, the system we have is rotating in the disk and the central object. The plasma is ionized. We have all the ingredient to have a complex structure involving rotation and magnetic field.

\section{Thermal Accretion and Winds}
\subsection{Spherical Accretion and Wind}
\subsubsection{Bondi Accretion}
\subsubsection{Parker wind or thermally driven outflow}
\subsection{More on 1D models}
\subsubsection{Critical surfaces and Horizons}
\subsubsection{De Laval Nozzle}
\subsubsection{Vertical structure of a Thin Keplerian Disk}

\section{Theory of Standard Accretion Disks}
\subsection{Shakura Sunyaev Standard Thin Disk}
\subsubsection{Thin Accretion Disk}
\subsubsection{Steady Accretion Disk}
\subsubsection{Self-consistency}
\subsection{Origin of Turbulence in Standard Thin Disks}
\subsection{Radiation Supported Thin Disk}
\subsection{Advection Dominated disks and slim disks}
\subsection{Magnetized Accretion disks}

\section{Theory of Magnetized Outflows}
\subsection{Theory of Magnetized Jets}
\subsubsection{Axisymmetric Steady Jets}
\subsubsection{Magneto Centrifugally Driven Wind}
\subsection{Acceleration and Collimation}
\subsubsection{Magnetic Acceleration}
\subsubsection{Thermal or Pressure Acceleration}
\subsubsection{Pressure of thermal Collimation}
\subsubsection{Magnetic Collimation}
\subsection{Steady vs Time dependent, Analytical vs Simulations}
\subsubsection{Analytical models are Steady}
\subsubsection{Numerical simulations are Time-dependent}

\section{Young Stars}
\section{Extra-Galactic Nuclei}
\section{X-ray binaries and microquasars}
\end{document}