\documentclass[10pt,a4paper,english,draft]{article}

% Document layout
%\usepackage[xetex,
\usepackage[a4paper,
            margin=2cm,
            headheight=0.4cm,
            headsep=0.8cm,
            footskip=1.2cm,
            nomarginpar,
            ]{geometry}

\setlength{\parindent}{0pt}
\setlength{\parskip}{1ex}

% Must be before mathspec
\usepackage{amsmath}
\usepackage{amsfonts}
%\usepackage{amssymb}  Not used currently

% For boxed equations
\usepackage{tcolorbox}
\tcbuselibrary{theorems}

% Fonts settings
%\usepackage{mathspec}
%\defaultfontfeatures{Ligatures=TeX,RawFeature={-calt,+tnum,+lnum}}
%\setmainfont[BoldFont={Adobe Garamond Pro Semibold},BoldItalicFont={Adobe Garamond Pro Semibold Italic}]{EB Garamond}
%\setmainfont{EB Garamond}
%\newfontfamily\NotoSans{Noto Sans}
\usepackage[utf8]{inputenc}
\usepackage[T1]{fontenc}
\usepackage{lmodern}

% [english] is set globaly above
\usepackage{babel}

% Define some colors
\usepackage{xcolor}
\definecolor{linkcolor}{rgb}{0, 0, 0.6}
\definecolor{brandeisblue}{rgb}{0, 0.44, 1.0}
\definecolor{capri}{rgb}{0, 0.75, 1.0}
\definecolor{dgreen}{rgb}{0.1, 0.6, 0.1}
\definecolor{caribbeangreen}{rgb}{0, 0.8, 0.6}

% And use them for sections titles
\usepackage{sectsty}
\sectionfont{\color{blue}}
\subsectionfont{\color{brandeisblue}}
\subsubsectionfont{\color{capri}}
\paragraphfont{\color{dgreen}}

% PDF properties
%\usepackage[xetex,
\usepackage[unicode=true,
            pdfstartview=FitV,
            colorlinks=true,
            citecolor=linkcolor,
            linkcolor=linkcolor,
            urlcolor=linkcolor,
            hyperindex=true,
            ]{hyperref}

\hypersetup{pdfauthor={Christophe Sauty},
            pdftitle={Accretion and Jets},
            pdfsubject={Paris -- M2 AAIS ET8 / Porto -- Doctorate},
            pdfkeywords={accretion, jets}}

% Custom header & footer
\usepackage{fancyhdr}
\pagestyle{fancy}
\fancyhead[L]{\scriptsize\textsc{Paris -- M2 AAIS ET8 / Porto -- Doctorate}}
\fancyhead[R]{\scriptsize\textsc{2015--2016 C. Sauty}}
\fancyfoot[C]{\thepage}

% Cross-link references
\usepackage[nameinlink]{cleveref}

% Figures
\usepackage{graphicx}
%\usepackage[abs]{overpic}
\graphicspath{{figures/}}

% Units display
\usepackage{siunitx}
\sisetup{
  number-unit-product = \,,
  inter-unit-product = \ensuremath{{}\cdot{}},
  separate-uncertainty = true,
  multi-part-units = single,
  range-phrase = \text{ -- }
}
\DeclareSIUnit\cgs{cgs}
\DeclareSIUnit\erg{erg}
\DeclareSIUnit\yr{yr}
\DeclareSIUnit\year{yr}
\DeclareSIUnit\pc{pc}
\DeclareSIUnit\kpc{kpc}
\DeclareSIUnit\Mpc{Mpc}

% Advanced tables
\usepackage{tabularx}

% References
\usepackage{natbib}

% Better inline fractions, derivatives
\usepackage{nicefrac}
\usepackage{esdiff}

% Package to insert fixmes
\usepackage{fixme}


% Fix spacing around \left and \right
\let\originalleft\left
\let\originalright\right
\renewcommand{\left}{\mathopen{}\mathclose\bgroup\originalleft}
\renewcommand{\right}{\aftergroup\egroup\originalright}

\newcommand\FIXME[1]{{\color{red}FIXME #1}}
\renewcommand\d{\mathrm{d}}        % the d of 'dx/dt'
\newcommand{\E}[1]{\cdot 10^{#1}}  % 10^{…} macro
\newcommand{\un}[1]{\ \mathrm{#1}} % for units

\title{\textsc{Accretion and Jets}}
\author{\textsc{Christophe Sauty}}
\date{\today}

\begin{document}

% First pages : i, ii, iii, iv, ...
\pagenumbering{roman}

\maketitle

\begin{center}
    $M_{\odot}$ %C'est ironique, non? -Un peu oui!
%    $M_\text{\NotoSans ☺}$
\end{center}

\tableofcontents

% Beginning of the proper content
\newpage
\pagenumbering{arabic} % 1, 2, 3, ...

\section{The necessity of the Accretion-Jet Mechanism}

\subsection{Active Galactic Nuclei}

\subsubsection{Radio quiet, Seyfert galaxies}

The first spectrum of a Seyfert galaxy was taken in 1943.

In the catalogue of 3C sources, it was first observed that there was a large
population of highly blue "stars" with unknown large lines (8000 km/s), like in
3C48 (Matthews et al., 1961, AAS, Dec1960). Moreover it was noticed that those
objects where redshifted. Greenstein and Schmidt, in 1964, for 3C48 and 3C273,
measured redshifts of $z=0.45$ and $0.16$. Thus those objects have to be at
cosmological distances but with very high luminosity.

Seyfert galaxies are usually spiral galaxies. They have a radio core, which
fluctuates in brightness. Because of the intense radio core they are classified
in the larger family of Active Galactic Nuclei. A long debate occurred
initially to know if the radio emission could be constructed form star burst
emission. Nowadays, the radio core and the whole spectrum (because we have
multi-wavelength observations) is understood as the result of the emission of
an accretion disk with or without outflows around a central supermassive black
hole.

As radio-quiet objects represent $\sim 90 \% $ of the 'average' AGN, they also
represent a very large source for the gamma ray background emission at low
level.

Illustrations: NGC 4258, Seyfert galaxy in Ursa major. Supermassive black hole
of 39 million Sun mass. Warped disk, two light years in diameter.

H2O masers amplify the microwave radio emission disk edge-on orbits 2 million
miles per hour.

\subsubsection{Radio Loud, Narrow Line Radio Galaxies}

Conversely to Seyfert galaxies, radio loud galaxies have an extended radio
emission in addition to the radio core. They are usually seen in giant
elliptical galaxies and not spiral ones. From our observational point of view,
the most powerful sources as such are of course quasars and BL Lac objects.
However the most striking structure is seen in radio galaxies of Fanaroff Riley
types.

Colour composite image of Centaurus A, revealing the lobes and jets emanating
from the central black hole of the galaxy. This is a composite of images
obtained with three instruments, operating at very different wavelengths.[...]

Radio loud sources and quasars have relativistic jets in addition to the
accretion disk, with bulk Lorentz factor $\Gamma=\dfrac{1}{\sqrt{1-V^2/c^2}}$
of 3 to 10 or possibly higher.

They are two types of radio loud galaxies, FRI and FRII.
\begin{itemize}
    \item FRI radio galaxies show elongated jets on both sides, on the kpc
          scale terminated with two radio lobes (spatially extended structure).
          On the pc scale, we see one-sides jets with relativistic speeds. On
          the kpc, the jet is sub-relativistic and has decelerated.
    \item FRII radio galaxies are one sided and relativistic on the pc and the
          kpc scales. The are more powerful and terminated with two hot spots.
          This implies that, although the visible jet is one sided, counter jet
          exist but is invisible because it is not relativistically beamed.
    \item HYMORS is a new class of FR galaxies, which are FRI on one side and
          FRII on the other side. This may be explained if the environment
          plays an important role.
\end{itemize}

\subsubsection{Quasars, BL Lac, QSO}

The other radio sources belong to the wider class of quasars (quasi-stellar
objects). They are radio quiet as well as radio loud quasars (QSOs). As for
radio galaxies, proper motions could be measured in some quasars with jets. In
all cases high relativistic velocities are deduced. Quasars, BL Lac and QSO
appear to be pointing towards us, which explains - because of the Doppler
effect - that they are more luminous than radio galaxies.

\subsection{Young Stellar Objects}

\subsubsection{Definition}

A \textbf{pre-main sequence star} (PMS star or PMS object) is a star that has
not yet reached the main sequence. In the following we shall consider mainly
loss mass stars with mass less or equal to two solar masses. High mass pre-main
sequence star exhibits strong outflows and winds but not well structure jets at
least at the present stage of observations. Loss mass stars go through the
stage of T Tauri star or FU Orionis star ($<2$ solar mass). High mass stars are
usually Herbig Ae/Be stars (2-8 solar mass).

\subsubsection{\texorpdfstring{HH objects and $CO$ Outflows}{HH objects and CO Outflows}}

Jets in YSOs where first seen through HH objects. They were defined as bright
knots initially. Usually aligned along a given axis. With the development of
new instruments (in particular the HST), they appear as complex structure and
often internal shocks or bow shocks. Bow shocks are characterized by the
C-Shock front as seen on the extremity of HH34 below (right).

In HH34 (and many other HH objects), the evolution seen from various successive
observations allows to measure proper motion of the knots. Typical values of a
few hundreds of km/s are usually inferred. However some knots may be standing
shocks and the proper motion of the shocks is not directly the speed of the
underlying jets though similar velocities are measured by other means.

These premain stars have accretion-ejection systems with typical accretion
rates of $\dot{M}_{accr}=\SI{e7}{M_\odot.yr^{-1}}$ to $\SI{e-6}{M_{\odot}.yr^{-1}}$.
Around $30\%$ of these objects also have jets with typical wind mass loss rates
of $\dot{M}_{jet}=\SI{e-8}{M_{\odot}.yr^{-1}}$ to $\SI{e-7}{M_{\odot}.yr^{-1}}$, densities
$n_{jet}=10^3$ to $\SI{e7}{cm^{-3}}$ (this is strongly model dependent) and
temperatures around $10^3$ to $\SI{e4}{K}$. Of course the density in the accretion
disk is much higher and the temperature much lower.

\subsubsection{\texorpdfstring{Molecular flows $H_2$, $CO$}{Molecular flows H2, CO}}

Next picture of the HST is called ``Birth and death of the Milky Way''. It shows
a multi-wavelength picture with dark clouds illuminated and ionized by stars.
Clouds are clumps resulting from the Parker instability in the Milky Way. They
are themselves unstable and because of gravitational collapse they contract
forming accretion torii and disks.

Some disks (seen in infra red, red contours in next figure) show very powerful
molecular outflows sometimes associated with the previous HH objects and
Optical jets but not always. Molecular outflows with mass loss rates of
$\dot{M}_{jet}=\SI{e-5}{M_{\odot}.yr^{-1}}$, velocity a few 10 km/s and cold
temperatures of a tens of Kelvin. The underlying accretion disks have accretion
rates higher than typically $\dot{M}_{acrr}=\SI{e-4}{M_{\odot}.yr^{-1}}$.

HH211: Another nice example of a powerful molecular outflow with its inner
accretion disk. Note that the inner structure is not very different from some
pictures of Planetary Nebulae.

Molecular outflows are collimated in the observational perspective but not as
well as optical jets. From a theoretical point of view the opening angle is so
large that they may be considered as uncollimated winds. The opening angle is
the angle the outflow makes from its central source. They are usually
surrounding the optical jet when there is one.

\subsubsection{Accretion in YSOs}

The accretion process is always present even if outflows are not. This means
either that the ejection is episodic or (more likely) that accretion does not
necessarily require all the time the formation of a collimated source.

Before having disk images, the presence of an accretion disk was deduced from
the infrared excess measured by the Spectral Index. \[\text{Spectral
Index:}\quad \alpha=\frac{d\log(\nu F)}{d \log(\nu)}\quad\text{where}\quad
F_{\nu}\quad\text{is the flux density.}\] For disks in YSO, the spectral index
is calculated in the wavelength interval of $2.2-\SI{10}{\um}$ (near infrared
region).

In 1986 Lada C.J. and Wilking B.A. were the first to established the existence
of various classes now known as Class I,II,III. In 1993 Andre et al. discovered
a younger class called class 0, with a strong sub-millimeter emission, but very
faint at $\lambda<\SI{10}{\um}$. In summary the classes are:
\begin{itemize}
    \item \textbf{Class 0} sources: undetectable at $\lambda<\SI{10}{\um}$
    \item \textbf{Class I} sources: $\alpha>0.3$
    \item \textbf{Flat spectrum} sources: $0.3>\alpha>-0.3$
    \item \textbf{Class II} sources: $-0.3>\alpha>-1.6$
    \item \textbf{Class III} sources: $\alpha<-1.6$
\end{itemize}

The following figure gives a schematic view of the spectrum and the role of the
accretion disk (from André, 1996).

More information on accretion disks is expected from next generation of
instruments like ALMA and Herschell (HIFI).

\subsection{Need for Accretion, Eddington Luminosity}

\subsubsection{Luminosity: Only gravity can do it}

Why do we need to have accretion ? After all the equilibrium of disk does not
need any accretion, it could rapidly be under the action of the centrifugal
force.

Accretion,
\begin{itemize}
    \item allows to increase the mass of the central object, either black hole
          or super massive black hole in AGN, or Young Stars. This is of course
          an essential ingredient in the process of star formation and Active
          Galactic Nuclei formation.
    \item is a way to convert gravitational energy into radiation. This was the
          first argument to invoke accretion in the context of cataclysmic
          variables.
    \item is a way to redistribute angular momentum:
          \begin{itemize}
              \item either because of viscosity. This is thought to be the case
                    in low magnetization disk where there is no outflow. The
                    low magnetization also favors MRI magneto rotational
                    instabilities and thus the presence of highly turbulent
                    accretion
              \item or because it induces a strong outflow. This would be the
                    case for highly magnetized disk where the
                    magnetocentrifugal driving can eject plasma. Magnetic
                    diffusivity is still needed however.
          \end{itemize}
    \item is a way to brake the central star. The braking of the central star
          was though to occur through the magnetosphere connection between the
          star and the disk. This is the so-called disk locking. In the case of
          YSO, it turns out to be inefficient and stellar winds are needed.
\end{itemize}

In order to quantify accretion we can estimate the specific available energy.
It corresponds to the free fall energy:

\begin{equation}
  \frac{E}{m} \approx \frac{GM_*}{R}
\end{equation}

The available power is:

\begin{align}
P & = \frac{dE}{dt} \approx \frac{GM_*}{R} \frac{dm}{dt} = \frac{GM_*}{R} \dot{M}_{acc}
    \\
  &\approx \num{e38} \left(\frac{\dot{M}_{acc}}{\SI{e18}{\g\per\s}}\right) \left(\frac{M}{\SI{2e33}{\g}}\right) \left(\frac{\SI{7e10}{\cm}}{R}\right) \si{\erg\per\s}
    \\
  &\approx 10^{38} \left(\frac{\dot{M}_{acc}}{\SI{e-8}{M\per\yr}}\right) \left(\frac{M}{M_{\odot}}\right) \left(\frac{R_{\odot}}{R}\right) \si{\erg\per\s}
\end{align}
with
\begin{align}
  M_{\odot} &= \SI{2e33}{g} \\
  R_{\odot} &= \SI{7e10}{\cm} \\
  R_{NS}   &= \SI{10}{\km} = \SI{e6}{\cm} = \SI{e-4}{R_\odot}
\end{align}

Thus the available power depends on the accretion rate and the compactness M/R. The efficiency is the conversion rate,
\begin{equation}
 \frac{dE}{dt} = \varepsilon \dot{M} c^{2} \Rightarrow \varepsilon = \frac{GM_*}{c^{2}R}
\end{equation}

In the Newtonian regime, we obtain an efficiency of $\varepsilon = 0.15$ for an object with a radius of 10 km, 1 $M_{\odot}$, typical for a neutron star or a non rotating galactic black hole.

In the relativistic regime for a black hole in maximum rotation we have a larger range of efficiency $\varepsilon \approx 0.059-0.42$

We can compare the efficiency in various fields in Physics :
\begin{itemize}
  \item In Chemistry the typical energy is $\approx \SI{13}{eV}$, energy of ionization, therefore $\varepsilon < 10^{-8}$ thus
  \begin{equation}
  \frac{dE}{dt} \approx \frac{\SI{1}{eV}}{m_{p}c^{2}} m_{p}c^{2}\dot{M}
  \end{equation}
  \item In nuclear fusion $\approx \SI{1}{MeV}$, $4H \rightarrow He$ , $\varepsilon < 10^{-2}$
  \begin{equation}
  \frac {\Delta E_{4H \rightarrow He}}{mc^{2}} \approx \dfrac{4m_{H}-m_{He}}{4m_{H}c^{2}} c^{2}\dot{M}
  \end{equation}
\end{itemize}

Accretion is ten times more efficient per mass unit to convert energy with a compact object.
\begin{itemize}
  \item If it is a galactic Black Hole we get
  \begin{equation}
  R_{S} =\frac{2GM_{BH}}{c^{2}} = \SI{3}{\km} \frac{M_{BH}}{R_{S}} = 10^5\frac{M_{\odot}}{R_{\odot}} , \varepsilon = 0.5
  \end{equation}
  \item This is similar for a Neutron Star
  \item For a Supermassive Black Hole the compactness and the efficiency are identical, only the radius differs
  \begin{equation}
  R_{S} =\frac{2GM_{BH}}{c^{2}} = \num{3e11}-\SI{e14}{cm}
  \end{equation}
  \item For a White Dwarf
  \begin{equation}
  \frac {M_{WD}}{R_{*}} =10^{2}\frac{M_{\odot}}{R_{\odot}} ,\quad M_{WD} = M_{\odot},\quad R_{*}=10^{-2}R_{\odot} = \SI{e9}{cm}
  \end{equation}
  \item For a Young Star
  \begin{equation}
  \frac {M_{YSO}}{R_{*}} \approx \frac{M_{*}}{R_{*}} ,\quad M_{YSO} = M_{\odot},\quad R_{*}=2R_{\odot} = \SI{2e11}{cm}
  \end{equation}
  \item For Jupiter
  \begin{equation}
  \frac {M_{J}}{R_{*}} = 10^{-2}\frac{M_{\odot}}{R_{\odot}} ,\quad M_{J} = \SI{e-3}{M_{\odot}},\quad R_{*}=\SI{0.1}{R_{\odot}} = \SI{e10}{cm}
  \end{equation}
\end{itemize}

\subsubsection{Eddington Luminosity}
We can define the Eddington luminosity, assuming isotropy and equilibrium between radiative pressure and gravity:
\begin{align}
  F_{grav} & = F_{rad} \\
  \frac{GMm_p}{R^2} &= P_{rad}\sigma_{T,\gamma/e^{-}} \\
           & = \frac{u_{rad}}{3}\sigma_T \\
           &= \frac{1}{3}\frac{LR}{\frac{4}{3}\pi R^3 c}\\
  \frac{GMm_p}{R^2} &= \frac{L\sigma_T}{4\pi R^2 c}
\end{align}
\fxnote{Missing picture}

We can rewrite the gravitational force:
\begin{equation}
  F_\mathrm{grav} = \frac{GMm_p}{R^2} = F_\mathrm{rad} = \sigma_T \frac{L}{4\pi R^2}\frac{1}{c}
\end{equation}

Thus, it defines a luminosity called \emph{Eddington Luminosity} where the radiative pressure gradient equals the gravitational force:
\begin{align}
  L_\mathrm{Edd} &\approx \num{e33} \left( \frac{M}{M_\odot} \right) \si{\erg\per\s} \\
  \dot{M}_\mathrm{Edd} &\approx \num{e18} \left( \frac{\varepsilon}{0.1} \right) \left( \frac{M}{M_\odot} \right) \si{\g\per\s} = \SI{e-8}{M\per\yr}
\end{align}

\subsubsection{The disk spectrum}
Gas heating implies a balance between matter and radiation. This can be expressed under the form of a black body such that:
\begin{equation}
  \frac{\d E}{\d t} = \frac{GM_\star}{R}\dot{M}_\mathrm{acc} = 4\pi R^2\sigma T^4
\end{equation}
and the effective temperature of the equivalent black body is given by:
\begin{align}
  k_b T_{bb} & \approx \SI{2}{kEv} \left( \frac{\dot{M}}{\SI{e-8}{M\per\yr}} \right)^{1/4} \left( \frac{M}{M_\odot} \right)^{1/4} \left( \frac{R}{\SI{e-4} R_\odot} \right)^{-3/4} \\
  k_b T_{bb} & \approx \SI{1.5}{keV} \left( \frac{L_\mathrm{Edd}}{\SI{e38}{\erg\per\s}} \right)^{1/4} \left( \frac{R}{\SI{e6}{\cm}} \right)^{-1/2}
\end{align}
Then if all the radiation is converted into energy:
\begin{align}
  \frac{3}{2}\dot{M}\frac{k_b T_{bb}}{\mu m_H} & \approx \frac{GM_\star}{R}\dot{M_\mathrm{acc}}
\end{align}
\begin{equation} \tcboxmath{
  \begin{split}
    \frac{3}{2}k_bT_{bb} & \approx \mu m_H \frac{GM_\star}{R} \\
                         & \approx\num{90} \left( \frac{M}{M_\odot} \right) \left( \frac{R}{\SI{e6}{cm}} \right)^{-1} \si{MeV}
  \end{split}
}
\end{equation}

\fxnote{it is highlighted in the original text}

We get for:
\begin{itemize}
\item \textbf{a galactic black hole}
  \begin{equation}
    \SI{2}{kEv}\left( \frac{\dot{M}}{\SI{e-8}{M_\odot\per\year}} \right)^{1/4} \left( \frac{M}{M_\odot} \right)^{1/4} \left( \frac{R}{\SI{e-4}{R_\odot}} \right)^{-3/4} < kT < \SI{90}{MEv}\left( \frac{M}{M_\odot} \right)^{1/4} \left( \frac{R}{\SI{e-4}{R_\odot}} \right)^{-1}
  \end{equation}
  For a mass loss rate of $\dot{M} = \SI{e-8}{M_\odot\per\year}$, the spectrum spans from soft X to gamma rays.
\item \textbf{a white dwarf}
  \begin{equation}
    \SI{10}{eV} < kT < \SI{45}{keV}
  \end{equation}
  The spectrum is from $\SI{1000}{\angstrom}$ to hard X rays.
\item \textbf{a $\SI{e8}{M_\odot}$ super-massive black hole} and an accretion rate $\dot{M} = 0.1-\SI{1}{M_\odot\per\yr}$,
  \begin{equation}
    \SI{20}{eV} < kT < \SI{90}{MeV}
  \end{equation}
  The spectrum is from UV to gamma, softer than stellar BHs
\item \textbf{a young low mass star} $M = M_\odot$, a mass loss rate of $\SIrange{e-7}{e-6}{M_\odot\per\year}$, the radiation spans :
  \begin{equation}
    0.3-\SI{3}{eV} < kT < \SI{4.5}{keV}
  \end{equation}
  corresponding to radiations from IR, visible to X rays.
\item \textbf{a Jupiter}, the range is 
  \begin{equation}
    \SI{e-3}{eV} < kT < \SI{5}{eV}
  \end{equation}
  from far IR to optical

\end{itemize}

As a matter of fact the range of spectrum we just deduced coincides with the observed spectrum for accretion disks around these objects.


\subsubsection{AGN Luminosity and Spectra}
The luminosity observed in AGNs is of the order of $\SI{e46}{\erg\per\s}$. If the total luminosity is lower than the Eddington luminosity, which is reasonable, the mass of the central black hole has to be larger than $M > \SI{e8}{M_\odot}$.

The luminosity could be due to nuclear reactions, if we have a large population of star burst. This implies a burst rate of $\dot{M} \sim \SI{10}{M_\odot\per\yr}$. Considering that the galaxy has $M_\mathrm{galaxy} \sim \SI{e11}{M_\odot} \sim \SI{e11}{stars}$, it would be emptied in 10 billion years. This is too short and excludes the scenario of AGN being star burst galaxies.

Conversely, if the luminosity is due to accretion, the accretion rate required is $\dot{M} \sim \si{M_\odot\per\yr}$. The mass luminosity ratio is $M/L \sim 10$. Thus the luminosity of a standard galaxy is $L_\mathrm{galaxy} \sim \SI{e43}{\erg\per\s}$ and the luminosity of a quasar is a thousand times higher : $L_\mathrm{quasar}\sim 1000 L_\mathrm{galaxy}$. \textbf{For Seyfert galaxies, the luminosity is typically around $\SIrange{e43}{e45}{\erg\per\s}$ while for quasars it is between $\SIrange{e45}{e48}{\erg\per\s}$}.

The observed variability is typically of one week $\Delta t \sim \SI{6e5}{s}$. Considering that the speed of light is the upper speed limit, the causality principle implies that the emission region cannot be larger than $R \leq c\Delta t \sim \SI{2e16}{cm} \sim \SI{0.01}{pc}$. This means that the emission comes from a very narrow region with a rather high density. To satisfy this, it is difficult to have stellar clusters (e.g. Sgr A : $\SI{4e16}{M_\odot\per\pc^3}$).

The figure bellow \fxnote{put the figure} displays a typical \textbf{BL Lac Spectral Distribution}. \textbf{BL Lac} are typical radio loud AGNs. Nowadays, we have a good knowledge of the multi-wavelength spectrum from radio to gamma rays.

\textbf{Seyfert and Quasar Spectral Distributions} are quite similar to BL Lac. Each component is explained by the composition of the disk and the jet. The central BH is surrounded by a cold accretion disk of gas and dust. On top of the disk, close to the central object, a hot corona must be present. The radio levels depend on the presence or of a central jet.
\fxnote{Missing figure here}

The cascade model explains how the different frequencies are emitted and how they are correlated in space and time as shown in the following figure.
\fxnote{missing figure here}
\subsection{Need for Jets}
\subsubsection{Energy in the extended lobes}
In radio loud galaxies and quasars, an extended emission into the lobes is observed. Such galaxies are at cosmological distances, typically $\SI{200}{\Mpc}$. As $d\sim z\frac{c }{H}$, we have a redshift of $z\sim 0.05$.

Considering that the lobe size is of the order of magnitude $d_\mathrm{lobe}\sim \SI{100}{\kpc}$, for a plasma that moves in the at speed $v\sim \SI{e8}{\cm\per\s} = \SI{1000}{\km\per\s}$, the time delay necessary to reach the lobe is $\Delta t_\mathrm{life} \sim \frac{\num{e5}\times\num{3e18}}{\num{e8}} \si{Myr} \sim \SI{100}{Myr}$. The observed luminosity begin $\SI{e44}{\erg\per\s}$ at $\SI{e9}{\Hz}$, the amount of total energy is huge and equal to $\num{e44}\times\SI{3e15}{\erg} \sim \SI{e60}{\erg}$. This is equivalent to \SI{e6}{M_\odot}.

We observe radio emission both in extended and compact sources. In extended sources, the radio spectrum goes like $F_v \sim v^{-0.5} \textrm{ or } v^{-1}$, while in compact sources it is $F_v \sim \mathrm{cst}$. The nature of the emission is not obvious and was in question for sometimes \fxnote{not really English there}:
\begin{itemize}
\item Thermal emission was immediately excluded for the radio component as it would have implied too high temperatures
\item \emph{Bremstrahlung} emission could have been a reasonable candidate, but because of absorption, one expects this process to be limited.
\item The observed high polarization could be the signature of a Compton emission but this would imply an extremely low temperature $T < \SI{3}{K}$.
\end{itemize}

Finally it turned out that \textbf{synchrotron} emission was the best candidate to explain the radio excess and the slope of the spectral emission.
\subsubsection{Synchrotron emission}
The total energy density in the lobes is
\begin{equation}
  U \sim \frac{10^{60}}{10^{69}} \approx \SI{e-9}{\erg\per\s}
\end{equation}
If we make a hypothesis of equipartition, which is completely arbitrary in the case of jets but the simplest assumption one can make, then $\frac{B^2}{8\pi} \sim \SI{e-9}{\erg\per\cm\cubed}$ and therefore $B \sim \SI{100}{\micro G}$. The synchrotron frequency is given by $\nu = \num{4.3e6} B\gamma^2 \sin\alpha\ \si{\Hz}$. At \SI{46}{Hz}, we have a Lorentz factors of $\gamma \sim \numrange{e3}{e4}$. This means that the sychrotron emission comes from relativistic electrons. The cooling time is of the order of \SI{1}{Myr}, which is much less than the life time. It has several consequences on the emission region:
\begin{itemize}
\item there must be an \emph{in situ} continuous production of highly relativistic electrons
\item only a jet can inject matter so far from the central core
\item there are shocks where electrons are re-accelerated
\end{itemize}

A large fraction of the total energy is not visible and must be in the form of kinetic energy. Only a jet that feeds the lobe can provide this energy, even if it is not visible.
\subsubsection{Proper motion}
Proper motion of knots is now measured in all the objects. These are the motion of the shocks so probably an underestimation of the jet speed itself. Nevertheless, the velocity is much higher than the local sonic speed and only jets can explain supersonic speeds. Typical values in YSO jets are a few hundreds of \si{\km\per\s}.

\fxnote{Missing figure of proper motion}

In AGN jets, we see apparent superluminal motions that can be explained only if the shocks and the jets are moving at (nearly) the speed of light. Superluminal motions are also observed in microquasars.
\fxnote{VLBA image + GRS 1915}
\subsection{Rotation and the Angular Momentum transfer}
\subsubsection{The Angular Momentum problems in YSO}
The biggest obstacle to the formation of stars from diffuse gas is that a star can contain only a tiny fraction (less than \SI{e18}{\cm\per\s}) of the initial angular momentum of the gas from which it formed (\SI{e23}{\cm\square\per\s} for a molecular cloud of scale \SI{1}{pc}, \SI{e23}{\cm\squared\per\s}), so that nearly all of this angular momentum must be removed or redistributed during the formation process. The angular momentum from the cloud can be removed by ambipolar diffusion. Yet the problem to go from the core to the star remains.

The specific angular momentum of typical star-forming molecular clouds are at least three orders of magnitude larger than the maximum speed at the equatorial plane of the star where the centrifugal force equals gravity. Thus, if we eguals the centrifugal force per unit mass and gravity, we get
\begin{equation}
  \frac{v^2_\textrm{break up}}{R_\star} = \frac{GM_\star}{R_\star^2}
  \Rightarrow v_\textrm{break up} = R_\star \Omega_\textrm{break up} = \sqrt{\frac{GM_\star}{R_\star}}
  \Rightarrow \Omega_\textrm{break up} = \sqrt{\frac{GM_\star}{R_\star^3}}
\end{equation}

For a typical T Tauri star, the break up value is one to a few hundreds \si{\km\per\s} while those stars rotate at around \SIrange{5}{20}{\km\per\s}. The stellar rotation is an order of magnitude bellow the break up velocity.

Here are some recent \textbf{Kepler} mission measurements of rotation in clusters.
\fxnote{Missing figure}
These results can be compared to Bouvier et al. results hereafter, showing plots of the evolution of the rotation as a function of time for Fast Rotators (blue lines) and Slow Rotators (red lines). The envelope velocity is always slower than the core velocity. The circle on the right is the position of our Sun. The curves correspond to evolutionary tracks and are model dependant.

\fxnote{Missing plots}

In the case of a Keplerian disk, to the first order, the centrifugal force balances gravity $\rho \frac{v_\phi^2}{r} \approx \rho \frac{GM_\star}{r^2}$. Thus, similarly to the break up velocity we get the Keplerian velocity and angular frequency:
\begin{equation}
  v_\phi = \sqrt{\frac{GM_\star}{r}}, \quad \Omega = \frac{v_\phi}{r} = \sqrt{\frac{GM_\star}{r^3}}
\end{equation}
\fxnote{Missing scheme}

If the disk edge is connected onto the star as on the above scheme, they should rotate at the same speed on the boundary layer and the star should thus rotate at its breakup velocity. As this is not the case, it is obvious that the star cannot be directly connected to the disk and that some mechanism should exist to evacuate the angular momentum. One solution to this problem is the disk-locking mechanism, where the angular momentum of the star is transferred to the disk \emph{via} the magnetosphere and then the viscosity of the disk removes the excess of angular momentum. Another solution is that the stellar wind itself carries away angular momentum while being braked by the magnetic field.
\subsubsection{Rotation of AGN and Microquasar Central Black Hole}
The problem of the transfer of angular momentum to the black hole in the accretion disk is similar to the one in YSO. All black holes are surely not maximally rotating, for instance the Galactic Black Hole of the disk luminosity shown on next figure is rotating at 0.6 its maximum angular velocity.

If the mass of a Kerr black hole is $M$, its angular momentum $J$, its Schwarzschild radius is
\begin{equation}
  R_s = \frac{2GM}{c^2}
\end{equation}
and the radius of the horizon is:
\begin{equation}
  R_\mathrm{Kerr} = \frac{Rs}{2} \left( 1 + \sqrt{1-\left( \frac{Jc}{GM^2} \right)^2 } \right)
\end{equation}
The angular momentum cannot be higher than the value that makes the square root negative, so
\begin{equation}
  J < J_\mathrm{max} = \frac{GM^2}{c} = MR_\mathrm{Kerr}c
\end{equation}
We define the rotation parameter as:
\begin{equation}
  \alpha = \frac{J}{J_\mathrm{max}} < 1
\end{equation}

How do we know the nature of the central black hole in AGNs? The main clue comes from the iron-K line which can be explained only with a central rotating black hole.

\fxnote{Missing figure}

The red component of the line is the signature of general relativity effects due to the BH. It is explained by the gravitational redshift (Einstein effect) in addition to special-relativity Doppler shift and beaming, as shown on the following figure of a synthetic Iron K Line produced by a rotating accretion disk around a BH.

\fxnote{Missing figure}

The line profile can be modelled using a Schwarzschild BH or a non rotating BH (below left). The discrepencies can be reduced by keeping the Schwarzschild BH, but taking into account the reflected continuum (below right).

\fxnote{Missing figure}

However, it seems the best fit is obtained by considering a rotating BH, or Kerr BH in maximal rotation.

The last figure is generated for a maximally rotating Kerr BH. Note the absence of the dotted line on the right part.
\subsubsection{Conclusion, The Global Picture of the Jet}

In conclusion, the system we have is a rotating disk and a central
object. The plasma is ionized. We have all the ingredients to have a complex
structure involving rotation and magnetic field.

\section{Thermal Accretion and Winds}
Because of rotation and magnetic fields, accretion and ejection are 3-dimensional and close to axisymmetry. However, some of the basic features can be under understood by looking a 1-dimensional processes and especially spherically-symmetric accretion and wind.
\subsection{Spherical Accretion and Wind}
\subsubsection{Bondi Accretion}
Let's consider an ideal fluid around a star in-falling (accretion), with a spherical symmetry. Using spherical coordinates $(r,\theta,\phi)$ we write the physical variables describing the fluid :
\begin{equation}
  \rho = \rho(r),\quad P = P(r), \quad \vec{v} = \vec{v}(r)
\end{equation}
The conservation of energy gives:
\begin{equation}
  E = \frac{1}{2}v^2 + h + \Phi_\mathrm{grav}
\end{equation}
For the sake of simplicity, let's assume the flow is isothermal. The gravitational potential is
\begin{equation}
  \Phi_\mathrm{grav} = - \frac{GM_\star}{r},\quad \textrm{$M_\star$ is the mass of the star}
\end{equation}
and the enthalpy is:
\begin{equation}
  h = \int \frac{\d p}{\rho} = c_s^2 \ln \rho
\end{equation}
because the flow is isothermal. Here, $c_s^2 = 2 \frac{k_b T}{m_H}$ is the sound speed.

The conservation equations are written as follow:
\begin{equation}
  \tcboxmath{
    \begin{split}
      \vec{\nabla}\cdot \left( \rho\vec{v} \right) = 0, \quad& \textrm{Mass conservation}\\
      \rho v \frac{\d v}{\d r} = -\frac{\d p }{\d r} - \rho \frac{GM_\star}{r^2},\quad & \textrm{Euler equation}\\
      p = c_s^2 \rho,\quad & \textrm{Equation of state}
    \end{split}
  }
\end{equation}
The mass conservation equation gives:
\begin{equation}
  \rho v r^2 = \textrm{constant} = \dot{m}, \quad \textrm{$\dot{m}$ is the accretion rate}
\end{equation}

Rewriting the energy, we get :
\begin{align}
  E &= \frac{1}{2}v^2 + c_s^2 \ln \frac{\dot{m}}{vr^2} - \frac{GM}{r} \\
    &= \frac{c_s^2}{2}\left[ \frac{v^2}{c_s^2} + 2\ln \frac{\dot{m}}{vr^2} - \frac{2GM}{c_s^2r}\right] \\
    &= \frac{c_s^2}{2} \left[\frac{v^2}{c_s^2} + \ln \frac{c_s^2}{v^2} + 4\ln \frac{2GM}{c_s^2r} - \frac{2GM}{c_s^2r} + 2\ln \frac{\dot{m}c_s}{(2GM)^2} + 2\ln c_s^2\right]
\end{align}
Let $r_c = \frac{GM}{2c_s^2}$. The Bernoulli constant gives that the solutions are given implicitly by drawing the following isocontours:
\begin{equation}
  \tcboxmath{
    \left( \frac{v}{c_s} \right)^2 - \ln \left( \frac{v}{c_s} \right)^2 - 4\ln \frac{r}{r_c} - 4 \frac{r_c }{r} = C
  }
\end{equation}
where $C$ is a constant.

By deriving this equation or directly from Euler equation,
\begin{align}
  \rho v \frac{\d v}{\d r}  &= -c_s^2 \frac{\d \rho}{\d r} - \rho \frac{GM}{r^2} \\
  r \frac{v}{c_s^2} \frac{\d v}{\d r} &= \underbrace{-\frac{r}{\rho}\frac{\d \rho}{\d r}}_\textrm{mass convervation} - \frac{GM}{c_s^2r} \\
                            &= r \left( \frac{2}{r} + \frac{c_s}{v} \frac{\d(v/c_s)}{\d r} \right) - \frac{2r_c }{r} \\
  \frac{r}{r_c }\frac{v}{c_s }\frac{\d (v/c_s)}{\d (r/r_c)} & = \frac{r}{r_c} \left( \frac{2r_c }{r} + \frac{c_s}{v} \frac{\d(v/c_s)}{\d(r/r_c )} \right) - \frac{2r_c }{r}
\end{align}

\begin{equation}
  \label{eq:speed_eq}
  \tcboxmath{
    \left( \frac{v}{c_s} - \frac{c_s}{v} \right) \frac{\d \left( \frac{v}{c_s} \right) }{\d \left( \frac{r}{r_c } \right) } = 2 \frac{r_c}{r} \left( 1 - \frac{r_c }{r} \right)
  }
\end{equation}

The set of solutions of the previous equation is displayed on the two following figures. This is the topology of the solutions.

We note $M = \frac{v}{c_s}$ the Mach number.

We could show that for polytropic winds ($p = Kp^\gamma$), the topology of the Mach number doesn't change even though the sound speed is no longer a constant but depends on distance :
\begin{equation}
  c_s^2 = \frac{\d p}{\d \rho}
\end{equation}

A few years before Parker, Bondi (1952) proposed that accretion onto a star or a black hole could become supersonic. This corresponds to one of the 2 critical solutions. Of course, supersonic accretion onto the star ends by a shock, which explains X-ray emissions of cataclysmic binaries.

We can calculate for this solution the mass accretion rate, which is by definition
\begin{equation}
  \dot{M} = 4\pi r^2 \rho (-v) = 4\pi r_c^2 \rho_s c_s
\end{equation}
Evaluating the energy at the critical point :
\begin{equation}
  E = \textrm{constant} = \frac{1}{2}v^2 + c_s^2 \ln\rho - \frac{GM}{r} = \frac{1}{2}c_s^2 + c_s^2 \ln \rho_s - \frac{GM}{r_c}
\end{equation}
we get the velocity both at infinity and close to the star :
\begin{align}
  v^2 &= 2c_s^2 \left( -\frac{3}{2} + \ln \frac{\rho_s}{\rho} \right)  + \frac{2GM}{r} \\
      & \xrightarrow{r\to 0} \frac{2GM}{r} \\
      & \xrightarrow{r\to\infty} 0, \Rightarrow \rho_S = \rho_\infty e^{3/2}
\end{align}

We see that close to the star, the fluid is in free fall. The mass accretion rate of the transsonic solution is thus:
\begin{equation}
  \dot{M}_\mathrm{transsonic} = \pi \left( GM \right)^2 e^{3/2} \frac{\rho_\infty}{c_s^3}
\end{equation}
\subsubsection{Parker wind or thermally driven outflow}
We now suppose that the ideal fluid around the star is out-flowing (wind). We assume spherical symmetry. Using the same spherical coordinates ($r, \theta, \phi$), we can derive the same equations with different boundary conditions. Here, conditions are given at the stellar surface and not at infinity. Again we have $\rho = \rho(r)$, $P = P(r)$, $\vec{v} = \vec{v}(r)$. 

Even if the fluid is magnetized, the assumption of spherical symmetry implied that the magnetic field parallel to the stream does not give any contribution. The Poynting flux is zero.

For the sake of simplicity, we'll assume again that the flow is isothermal. The original Parker model for the solar wind already contained all the necessary ingredients ; it assumed the flow to be isothermal close to the sun and adiabatic farther away where the wind is accelerated.

The final set of equation is the same. Basically we solve:
\begin{equation}
  \tcboxmath{
    \left( \frac{v}{c_s} - \frac{c_s}{v} \right) \frac{\d \left( \frac{v}{c_s} \right) }{\d \left( \frac{r}{r_c } \right) } = 2 \frac{r_c}{r} \left( 1 - \frac{r_c }{r} \right)\quad \textrm{or} \quad
    \left( \frac{v}{c_s} \right)^2 - \ln \left( \frac{v}{c_s} \right)^2 - 4\ln \frac{r}{r_c }- 4 \frac{r_c}{r} = C
  }
\end{equation}
The topology of the solutions is the same as previously but with a new interpretation. We can also plot in 3D the energy versus distance and velocity to make clear the critical point is sitting on a saddle. We still have $M = \frac{v}{c_s}$ the Mach number. We see that the r.h.s. of \eqref{eq:speed_eq} vanishes for $r=r_c$ and the l.h.s. for $v=v_s$ except if the flow has reached super sonic speed at the critical radius. Thus solutions have an horizontal or a vertical tangent except for the two critical solutions where there is \textbf{a fixed or critical point}. This is a \textbf{saddle} point\footnote{The two slopes of the critical solutions at the critical point can be determined using l'Hospital's rule, i.e. expanding $\frac{v}{c_s} = 1 + \delta$ and $\frac{r}{r_c} = 1 + \varepsilon$ to first order. We obtain a second algebraic equation for the slope $\delta/\varepsilon$ which solutions are the slopes at the saddle point,
\begin{equation*}
  \frac{\delta}{\varepsilon} = \frac{\d (v/c_s)}{\d (r/r_c)} = \frac{2 \frac{r_c }{r}\left( 1 - \frac{r_c }{r} \right) }{\frac{v}{c_s} - \frac{c_s}{v}} = \frac{2 \frac{1}{1+\varepsilon} \left( 1 - \frac{1}{1+\varepsilon} \right) }{1 + \delta - \frac{1}{1+\delta}} \approx \frac{2(1-\varepsilon)\varepsilon}{1 + \delta - 1 + \delta} \approx \frac{2\varepsilon}{2\delta} \Rightarrow \frac{\delta}{\varepsilon} = \pm 1
\end{equation*}} with two slopes. The saddle nature of the point is better seen in the 3D plot of the energy in phase space ($r, v$).

\subsubsection{Solar wind (Parker 1958)}
The breeze solutions have zero velocity asymptotically, but the pressure is non zero and cannot match the interstellar pressure :
\begin{equation}
  v^2 = 2 c_s^2 \left( -\frac{3}{2} + \ln \frac{\rho_s}{\rho} \right) + \frac{2GM}{r} \xrightarrow{r\to\infty} 0,\Rightarrow \rho_\infty = \rho_s e^{-3/2} \neq 0 \Rightarrow \rho_\infty \neq 0
\end{equation}
Thus they are very similar in behavior to the static solutions, which we have shown to be excluded. Parker showed that the only valid solution is the critical solution that becomes super sonic with zero pressure at infinity. It has then been confirmed observationnaly. Parker also showed studying the polytropic wind that acceleration can be obtained only if the polytropic index is $\gamma < \frac{3}{2}$. $\frac{c_p}{c_s} = \frac{5}{3}$ being the adiabatic value for a monoatomic gas, this means that the exists only because there is enough extended heating at the base of the solar corona.

There are some breeze solutions with heat conduction, which have zero pressure at infinity. Thus the debate was really strong during the 60s, until observations confirmed Parker's theory. Nowadays, the problem has slightly shifted. In fact for a temperature of one million Kelvin, the speed at the earth orbit is of the order of \SI{500}{\km\per\s} with this model. From the measurements of Ulysses spacecraft, we know that this is effectively the speed of the slow equatorial wind, but this model is not sufficient to explain the \SI{800}{\km\per\s} of the fast polar wind. Thus, series of models have been developed which involve sophisticated not-so-well-known coronal heating, kinetic theories, etc... is still an open debate.

Similarly to the accretion rate, one can calculate the mass loss rate from the star through the wind:
\begin{align}
  \dot{M} & = 4\pi r^2\rho v = 4\pi r^2_c\rho_sc_s \\
  \dot{M}_\odot & = \SI{e-14}{M_\odot\per\s} \\
  \dot{M}_\mathrm{TTauri} &= \SI{e-8}{M_\odot\per\s}
\end{align}

\subsection{More on 1D models}
\subsubsection{Critical surfaces and Horizons}

The critical point corresponds to sonic waves . The stellar surface losses mass in a steady
way only if the mass flux  adapts itself to cross the critical surface. The sonic waves going
upstream will make the ejection to evolve until it reaches the steady state. This critical suface
is analogous to the event horizon around black holes. Beyons this surface, the wind is unable
to propagate sound waves back to the source.

There is a strong analogy between the soncic surface and apparent horizon in black holes, 
except that the sound speed  replaces the speed of light. This is illustrated in the following
figures. 

\fxnote{Missing figure}

If the star rotates, in addition to the sonic surface/horizon there is also an accoustic ergosphere.

  
\subsubsection{De Laval Nozzle}

There is a strong analogy between the parker wind and a flow through a nozzle. A nozzle is a 
pipe with a variable cross section. It has a minimum and this minimum should coincide with the
sonic point to have strong supersonic flow,

\fxnote{Missing Figure}

Mass conservation : 

\begin{equation}
\vec{\nabla}.(\rho\vec{v}) \Rightarrow \int \rho\vec{v}.d\vec{S} = cst \Rightarrow \rho vS = cst 
\Rightarrow \frac{d\rho}{\rho} + \frac{d\nu}{\nu} + \frac{dS}{S} = 0
\end{equation}

Euler equation : \begin{equation} 
\rho v \frac{dv}{dx} = -\frac{dp}{dx} 
\end{equation}
Equation of state (energy) :
\begin{equation}
p(x) = c_{s}^2\rho(x)
\end{equation}  
 It follows : 
\begin{equation}
v\frac{dv}{dx} = -c_s^2\frac{d\rho}{\rho dx} = c_s^2\bigg(\frac{dv}{vdx}+\frac{dS}{Sdx}\bigg)
\Rightarrow \bigg( v - \frac{c_s^2}{v} \bigg)\frac{dv}{dx} = c_s^2\frac{dS}{Sdx}
\end{equation}
  
The flow can be transonic only if $\frac{dS}{Sdx} = 0 $ which means that the cross section
has an extremum. It is by default a minimum as at the entrance of the nozzle on the left,  $v<c_s$ and
$\frac{dv}{dx} > 0$ thus $\frac{dS}{dx} > 0 $. If this minimum exists, to be transsonic the flow must
attain the sound speed at the minimum cross section.
  
\paragraph{Spherical flow, Parker wind}
  
For a spherical wind we have : 
\begin{equation}
\bigg( v - \frac{c_s^2}{v}  \bigg)\frac{dv}{dr} = c_s^2\frac{dS}{Sdr} 
= \frac{2c_s^2}{r} - \frac{GM}{r^2} \Rightarrow \frac{d \ln S}{dr} = 
\frac{2}{r} - \frac{GM}{c_s^2 r2}   
\Rightarrow \ln S = \ln r^2 +\frac{GM}{c_s^2r} + k
\end{equation}

And finally : 
\begin{equation}
 \tcboxmath{S = Cr^2 \exp\bigg(\frac{GM}{c_s^2 r}\bigg)}
\end{equation}
 
 Gravity plays the role of reducing the effective cross sectionas te radial flux tubes expands 
 as $r^2$. There is no real throat just an effect of graviy.
  
\paragraph{Funnel ans 1D flow}
The analogy has been used to explain jets both in the YSO  and the AGN contexts. The argument 
is to use the presence of the surrounding thick disk close to the central object has shaped in 
the form a funnel.

This is the case of the famous \textbf{ Exhaust Model for AGN Jets} suggested by 
Blandford  \&  Rees 1974 (see correct figure), used also in the Ion Supported Torus
model of jet funnel by Phinney 1999 (see other correct figure).

\fxnote{Missing Figure}

Raga and Canto also used this model in 1989 for YSO, invoking a stellar wind to make 
the cavity.

\fxnote{Missing figure}

These models suffer strong flaws. The shape of the funnel is somewhat given and 
not the result of a really consistent resolution of the MHD equations. The strongest argument
is however that the throat of the funnel is always too far from the central object to maintain 
a very small opening angle for the jet.

This leads us to next chapterand to consider axisymmetric models ans not only 1D solutions.



 
\subsubsection{Vertical structure of a Thin Keplerian Disk}

\section{Theory of Standard Accretion Disks}

\subsection{Shakura Sunyaev Standard Thin Disk}

\subsubsection{Thin Accretion Disk}

\FIXME{Missing figure}

We assume a thin disk as previously Keplerian to first order where the height
can be neglected. In other words we integrate all equations on an annulus $R$
over the height of the disk $H(R)$. First we write mass conservation,
\begin{equation*}
    \diffp{\rho}{t} + \vec{\nabla}.(\rho \vec{v}) = 0
\end{equation*}

Introducing the surface mass density defined as
\begin{equation*}
    \Sigma = \int_{-H}^H \rho \d{z} \approx 2 \rho H
\end{equation*}

so that $2 \pi R \Sigma$ is the mass density per unit length. Integrating the
mass conservation over H (the integral and the derivative being linear and here
independant operators, they can be inverted),
\begin{align*}
    \int_{-H}^H \left( \diffp{\rho}{t} + \vec{\nabla}.(\rho \vec{v}) \right) = 0 % Not a mistake, just for code readability
    &= \diffp{\int_{-H}^H \rho \d{z}}{t} + \vec{\nabla}.\left( \int_{-H}^H \rho \vec{v} \d{z} \right) \\
    &= \diffp{\int_{-H}^H \rho \d{z}}{t} + \vec{\nabla}.\left( \vec{v} \int_{-H}^H \rho \d{z} \right) \\
    &= \diffp{\Sigma z}{t} + \vec{\nabla}.(\Sigma V_R)
\end{align*}

we get mass conservation on each annulus:
\begin{equation}
    \tcboxmath{
        \diffp{\Sigma}{t} + \frac{1}{R} \diffp{(R \Sigma V_R)}{R} = 0
    }
\end{equation}

We assume quasi-Keplerian rotation $V_\varphi = R \Omega_K = \sqrt{\frac{G
M}{R}}$. In reality rotation is slightly sub-Keplerian to allow accretion.

We then use the Angular Momentum conservation, which comes from the global
momentum equation,
\begin{equation*}
    \rho \diffp{\vec{v}}{t} + \rho (\vec{v}.\vec{\nabla})\vec{v} = \rho \vec{g} - \vec{\nabla} p - \vec{\nabla}.\left(\rho \overline{\vec{v'} \otimes \vec{v'}}\right) \quad \text{with} \quad V_\varphi \leq R \Omega_K
\end{equation*}

Momentum equation can be written with a turbulent pressure tensor
\begin{equation*}
    \diffp{}{t}(\rho \vec{v}) + \vec{\nabla}.\left(\rho \vec{v} \otimes \vec{v} + \tens{P}_{turb}\right) = \rho \vec{g}
\end{equation*}

\FIXME{Need of a better tensorial notation for Pturb}

with the turbulent pressure being
\begin{align*}
    \tens{P}_{turb} &= p_{gas} \tens{I} + \rho \overline{\vec{v'} \otimes \vec{v'}} \\
    \left(\overline{\vec{v'} \otimes \vec{v'}}\right)_{ij} &= \overline{v_i' v_j'} = - \nu_T \left( \diffp{\overline{v_i}}{{x_j}} + \diffp{\overline{v_j}}{{x_i}} \right) \\
    \Leftrightarrow \left(\overline{\vec{v'} \otimes \vec{v'}}\right) &= - \nu_T \left(\vec{\nabla} \otimes \vec{v} + \vec{v} \otimes \vec{\nabla}\right)
\end{align*}

\FIXME{Same as above}

The molecular viscosity is completely inefficient and cannot explain the source
of viscosity essential to have accretion,
\begin{equation*}
    \mathfrak{R}_\mathrm{e} = \frac{L^2}{\nu T} = \frac{L V}{\nu} \approx \num{e11}
\end{equation*}

Shakura \& Sunyaev suggested then (1971) that the source of viscosity is
turbulence giving an anomalous coefficient for the viscosity.
\begin{equation*}
    \mathfrak{R}_\mathrm{eT} = \frac{L^2}{\nu_T T} = \frac{L V}{\nu_T} \approx \num{e4}
\end{equation*}

\FIXME{Make it a reference}

To explain this turbulence one needs the disk to develop instabilities. Since
Balbus \& Hawley (1991), the instability most favorable seems to be the
Magneto-Rotational-Instabililty (MRI hereafter).

\FIXME{Make it a reference}

To get the angular momentum conservation on $\phi$, we can multiply by $\vec{R} \times$, and integrate over height,
\begin{align*}
    &\int_{-H}^H \vec{R} \times \diffp{}{t}(\rho \vec{v}) \d{z} + \int_{-H}^H \vec{R} \times \vec{\nabla}.\left(\rho \vec{v} \otimes \vec{v} + \tens{P}_{turb}\right) \d{z}
    = \\
    &\Rightarrow =
\end{align*}


\subsubsection{Steady Accretion Disk}
\subsubsection{Self-consistency}
\subsection{Origin of Turbulence in Standard Thin Disks}
turbulence comes from instabilities, For a weakly magnetized disk, magneto-rotational instability is a good candidate.
This instability was first studied bu Balbus and Hawley  (1991) in the context of accretion disk, though the
magneto-rotational was already formally described earlier by Russain groups. in the context of disks, 
it is a rather important instability because it may explain the origin of turbulence and allows to estimate 
the turbulent vicosity  and the turbulent magnetic diffusity in those objects. 

\FIXME{fix references in next paragraph}

The system is an accretion disk around a star as seen on Fig.XXX below, thread by a vertical homogeneous field
$\vec{B_o} = B_z\vec{e_z}$ such that the Lorents force is zero, using cylindrical coordinates $[r,\theta,\phi]$.
Thus gravity is balanced only by the centrifugal force. The disk rotates  at the Keplerian speed. The angular 
velocity is thus $\Omega = \frac{v_{\phi}}{r} = \sqrt{\frac{Gm_*}{r^3}}$  (see III.2.2.3). The magnetic fiels 
is frozen  in the plasma. it is easy to verify that the induction equation is trivially satisfied.

\FIXME{Missing figure}

In the initial non pertubed state, let consider three distanes sorted as follow $r_1 < r_0 < r_2 $. At 
each point, gravity balances the centrifugal fore  such that we can writte : 
\begin{equation}
\underbrace{\frac{Gm}{r_1^2} = \frac{r_1^2\Omega_1^2}{r_1}}_{\text{at positon  } r_1} > 
\underbrace{\frac{Gm}{r_0^2} = \frac{r_0^2\Omega_0^2}{r_0}}_{\text{at position }r_0 > r_1}	>
\underbrace{\frac{Gm}{r_2^2} = \frac{r_2^2\Omega_2^2}{r_2}}_{\text{at position }r_2 > r_0}	
\end{equation}

This just expresses that gravity decreases outward, thus the rotational speed as well. The gravitational 
force pushing inward (i.e $-\vec{e_r}$), and the centrifugal force  outward as it can be seen in the vector
form of the force balance :
\begin{equation}
-\rho\frac{Gm}{r^2}\vec{e_r} + \rho\frac{v_{\phi}^2}{r}\vec{e_r} = \vec{0}
\end{equation}

Now  on b) the system is perturbed. Point 1 comes from $r_0$ to $r_1$ and point 2 from 
$r_0$ to $r_2$. The key point here, contrarily to the centrifugal instability explaines in 
IV.3.3, is the flux freezing . It implies thath for sufficiently strong vertial fields this is not 
the angular momentum $ L$ which is conserved because it can be exchanged with the magnetic field,
but the angular velocity $\Omega$. Thus point 1 is pushed inwards toward the star where 
gravity is stronger but keeping its initial angular speed $\Omega_0$. Thus we immediatly get 
for point 1 : 
\begin{equation}
f_{gravity,1} = \frac{Gm}{r_1^2} = \frac{r_1^2\Omega_1^2	}{r_1} > 
\frac{r_1^2\Omega_0^2}{r_1} = f_{centrifugal,1}
\end{equation}

There, gravity is larger than the centrifugal force and pushes 1 further  down toward the centre. 
Conversely point 2 is pushed further out because there the opposite holds : 

\begin{equation}
f_{gravity,2} = \frac{Gm}{r_2^2} = \frac{r_2^2\Omega_2^2	}{r_2} < 
\frac{r_2^2\Omega_0^2}{r_2} = f_{centrifugal,2}
\end{equation}

\FIXME{Missing figure}

The end result is that the system is unstable and the perturbation develops. Of course the magnetic
tension of the poloidal field will limit the growth of the instability. This is why the unstability is usually
quoted to be efficient in weakly magnetized disks

A more quantitative analysis shows that the system  is unstable if : 
\begin{equation}
\tcboxmath{\frac{d\Omega^2}{dr} < 0 \Rightarrow \text{unstable}}
\end{equation}

Thus, the rotation in a star is usuallly such that the star is not affected, while Keplerian
disk are likely to be highly unstable  to this unstability. This unstability as a source of 
turbulence gives a  nice ans natural explanation of the alpha vicosity  used in the standard
Shakura-Sunyaev thin accretion disk model. In such a case,  the origin of turbulence is also 
the origin of the magnetic diffusity and the magnetic Prandtl number (The ratio of the viscosity to magnetic
diffusity) should be close to one. In which case, the magnetic field cannot be accreted with matter because it
diffuses through the lines. However it may still be accreted if angular momentum is removed not by viscosity
but by an outflowing  magneto-centrifugally driven disk wind. For those interested, the original paper by Balbus
and Hawley, ApJ, 1991, is easy to read anf gives details on both the qualitative as well as the linear 
analysis of the instability.


\subsection{Radiation Supported Thin Disk}
Now that we have explored the possibility of a thin accretion disk, let's calculate the thickness 
of such a disk provided taht it is entirely supported by radiation from the disk itself.

We use the following notation for the mass density and the temperature on the equatorial  plane : 

\begin{align}
	& \rho_c = \rho(z=0) = \frac{\Sigma}{2H}
	& T-c = T(z=0)
\end{align}
\subsection{Advection Dominated disks and slim disks}
\subsection{Magnetized Accretion disks}

\section{Theory of Magnetized Outflows}
\subsection{Theory of Magnetized Jets}
\subsubsection{Axisymmetric Steady Jets}
\subsubsection{Magneto Centrifugally Driven Wind}
\subsection{Acceleration and Collimation}
\subsubsection{Magnetic Acceleration}
\FIXME{mettre references correctess}

Magnetic acceleration can take two forms as illustrated. Either the magnetic is strongly
azimuthal (roped) and then uncoils itself as a spring. This corresponds to Fig (Fig XXX)
and the plasma gun model where this is the uncoiling of the magnetic spring (magnetic pressure gradient)
which provokes the acceleration. The second possibility (Fig XXX) is the Blandford Payne mechanism, where the magnetic field
is strong enough to maintain the plasma frozen at least up to the Alfven surface. As it rotates the effect is similar to a bead
on a wire and if the line is sufficiently inclined the plasmas is accelerated.

\FIXME{missing figure}

Fig. a) Plasma Gun mechanism is the uncoiling spring. this acceleration has been observed in lasers experiments using z pinch configuration, Unfortunately, it usually implies large initial velocitiy.
Moreover, the strong toroidal field is the most unstable situation.

Fig b). The poynting flux is converted into kinetic acceleration. This mechanism needs strong pressure 
in order to have  a large mass loading, and it is not available on the axis where rotation ceased. 

\subsubsection{Thermal or Pressure Acceleration}
\FIXME{missing figure}

Another way to obtain strongly accelerated jet is to invoke the presence of pressure gradients 
like in the solar wind. A purely thermally driven wind gives temperatures too high.

\begin{align*}
  & T \sim 10V^2 \\
  &  10^7 K \leftrightarrow	  \SI{1000}{\km\per \s } \\ 
  & 10^6 K \leftrightarrow \SI{20}{\km \per\s}  \\
  &10 ^{12-13} K \leftrightarrow  \SI{300000}{\km \per \s} \\ 
  & \Gamma \sim 3 \text{ - } 10
\end{align*}

However to solve this problem, one can invoke the fact that as in the solar wind where the 
effective temperature is ten times higher than the kinetic wind pressure, we may also argue
for jets that in addition to the kinetic temperature, the total pressure includes Ram, Alfven
or other wave pressure.


All this is consistant if the variation $\sim$ magnitude are reasonable.

\begin{equation}
P_{ram} \sim \frac{\delta V^2}{V^2} \qquad
P_{Alfven}	\sim \frac{\delta B^2}{b^2} \qquad
\frac{\delta V}{V } \text{ or } \frac{\delta B}{B} \sim 1
\end{equation}
\subsubsection{Pressure of thermal Collimation}
Collimtation can be the result of externa pressure onto the jet.

\FIXME{Missing figure}

Such collimation is not efficient on long distances from the soures, and we need help from the magnetic field.

\subsubsection{Magnetic Collimation}
Because of rotation, the build-up of azimuthal magnetic field creates a pinching or so-called "hoop=stress"

\FIXME{Missing figure}

Basically the Lorentz force is given at large distances from the source by the curvature R of the line.
\begin{equation}
j \times B = \frac{B^2_{\phi}}{R}
\end{equation}

Usually the hoop-stress gives a TOO efficient collimation and a very narrow jet, especially if there is only a disk wind. 
This is in favor of having a strong turbulent inner spine jet. This poses the electric current closure problem. In fact,
the electric current should close. it can close either out of the jet or more likely  on the axis. This demands again a 
two component jets with a spine component  from the central part and outer disk wind. 
\subsection{Steady vs Time dependent, Analytical vs Simulations}

\subsubsection{Analytical models are Steady}
The are different analytical models:
\begin{itemize}
\item \textbf{Meridional Self Similar} models have a dipolar magnetic flux, $A \propto \sin^2\theta$ They are necessarily thermally or pressure driven but well adapted to describe the jet axis and are usually an expansion with colatitiude of the equations.
\item \textbf{X-Wind} the come from the intermediate zone between the disk and the star. They are magnetocentrifugally driven. The disk should produce a fan wind which seems rather unstable.
 \item \textbf{Radial Self Similar} models use power laws, in particular for the magnetic flux $A \propto r^{-5/4}$. They are adapted for the external part of the jet and especially for Keplerian Disk Winds. They correspond to the case where boundaries can be neglected (in particular the inner and outer parts of the disk).

\fxnote{Missing figure}
\end{itemize}
\subsubsection{Numerical simulations are Time-dependent}
Many different AMR MHD codes are available:
\begin{itemize}
\item AMRVAC (Toth 1996; Nool \& Keppens 2002)
\item BATSRUS (Powell et al. 1999)
\item RIEMANN (Balsara 2000)
\item FLASH (Fryxell et al. 2002)
\item Nirvana (Zeigler 2005)
\item RAMSES (Fromang et al. 2006)
\item PLUTO (Mignone et al. 2007)
\item AstroBEAR (Cunnigham et al. 2008)
\item ZEUS, ATHENA (Stone et al., 199?)
\end{itemize}
To quote Stone, ``we use Athena which implements slightly different algorithms from all of these. These differences can be important. Diversity of methods is good.''.

\paragraph{Steady solution:}

In order to obtain real steady solution, self similarity is a necessary hypothesis so far. This might be a problem with the boundary conditions and the ersatz for scaling as they are not always compatible, realistically.

\paragraph{Time dependant solutions:}

Numerical simulations cannot last forever because they empty the disk. They also use large numerical diffusivity and have various problems in scaling.

Here after are a few examples of such simulations showing the collation by magnetic hoop stress.

\fxnote{Missing figures}
Time dependant schemes are also good to study the origin of the instabilities, the effects of star or magnetospheric variability and the origin of knots and shocks.

\fxnote{Missing figures}

Here is an example of a Kelvin Helmholtz instability in a simulation of a double component AGN relativistic jet using AMRVAC, by Meliani et al.

\fxnote{missing nice figure here}

\section{Young Stars}

\subsection{Disk classification}
The first classification of low mass YSOs was based on their spectral index:
\begin{equation}
  \alpha = \frac{\d \log (\nu F)}{\d \log \nu}
\end{equation}
where $F_\nu$ is the flux density.
\fxnote{Missing $\lambda = f(\lambda F_\lambda)$}
This spectral index is computed in the wavelength interval of \SIrange{2.2}{10}{\mu\m} (near infrared region). The IR excess is interpreted as the emission of the cold gas of the disk .Thus the index indicates how big is the IR excess and the importance of the disk in the spectrum.

\fxnote{Missing figures here}

Following Lad C.J. and Wiling B.A. (1986) and Andre et al. (1993), we split the sources in four classes:
\begin{itemize}
\item \textbf{Class 0} sources, undetectable at $\lambda < \SI{10}{\mu\m}$
\item \textbf{Class I} sources have $\alpha < 0.3$
\item \textbf{Flat spectrum} sources have $0.3 > \alpha > -0.3$
\item \textbf{Class II} sources have $-0.3 > \alpha > -1.6$
\item \textbf{Class III} sources have $\alpha < -1.6$
\end{itemize}

The accretion rate is not continuous during the life of the YS and various episodes are probably observed as Fu Ori and EXor outburst in the early stages (Class I phase or Class II).

\fxnote{Missing figure}

\subsection{Outflow classification}

This classifications obtained for disks can be associated to our present knowledge of outflows.
\fxnote{Missing figure}
\begin{itemize}
\item Class 0, disk driven jet -- not star ? The stellar core is there but the star may not be formed yet and brightening. Circulation models or outer disk driven winds are most likely
\item Class I, disk driven jet. Keplerian inner disks play a major role in magentocentrifugally driving the jet
\item Class II, classical T-Tauri YSO jet. Depending on the power of the jet, the inner Keplerian disk, the star/disk boundary or the star itself may drive the jet.
\item Class III, YSO weak (line) T-Tauri stars may drive a stellar wind (as main sequence stars) or an invisible jet. The (gas) disk is no longer connected to the star but may still be present
\item Main sequence stars drive stellar winds
\end{itemize}

\subsection{The classification as a signature of the evolution}

\subsubsection{Class 0}

Class 0 objects are characterized mostly by very powerful CO flows. Another nice example of a powerful molecular outflow, with its inner accretion disk. Note that the inner structure is no very different from some pictures of planetary nebulae.

\fxnote{Missing picture}

As illustrated below, there is an ``onion'' structure with a high velocity inner component (even in the submillimiter range not necessarily in optical) along the jet axis and a low component around.
\fxnote{Missing picture}

\subsubsection{Class I}

Nice images of the HH objects from the HST are coming from Class I objects where the star is still quite embedded.
\fxnote{Missing pictures}

The images of HH34, HH30, HH47 give a nice view at different scales. HH30 clearly shows the disk seen through illumination by the star. It gives this typical ``concave lens'' aspect, which may look like a Keplerian disk but this is not a proof (see \autoref{sec:counter_rotation}). This is the result of the central star illuminating the disk (these are class I objects).

Although the picture is not clear, HH30 has knots separated by a distance of a hundred \si{AU} which for a speed of a few \SI{100}{\km\per\s} corresponds to a time scale of \SI{1}{\year}. HH34 show knots every \SI{1000}{AU} corresponding to \SI{10}{\year}. (and also smaller scales of \SI{100}{AU} closer to the star). In fact, these time scales correspond to typical stellar cycles. On very large scales (HH47) the jet structure is more chaotic. This this last case, the central star is also more embedded.

In order to study the star variability with PLUTO, Matsakos et al. (2009) used variable injections of mass flux, to mimic the variability on \SI{10}{\year} (left side) and \SI{1}{\year} (right side). \SI{10}{\year} correspond to \SI{1000}{AU} which is compatible with the spacing of the knots in HH30, while a variability of \SI{1}{\year} would give a spacing of \SI{100}{AU} as seen in HH34. We see that if the stellar mass flux varies, shocks form and affect the whole jet, including the disk wind.
\fxnote{Missing picture}

Disks of Class I objects (seen in infra red, red contours in next figure) show very powerful molecular outflow associated with the previous HH objects and optical jets. Molecular outflows with mass loss rates of $\dot{M}_\mathrm{jet} = \SI{e-5}{M_\otimes\per\year}$, velocity of a few \SI{10}{\km\per\s} and cold temperatures of a few tens of Kelvin. The underlying accretion disk has accretion rates higher, typically $\dot{M}_\mathrm{accr} = \SI{e-4}{M_\otimes\per\year}$.
\fxnote{Missing figure}

\subsubsection{Class II, Classical T-Tauri stars}

\subsubsection{Class III, Weak Line T-Tauri stars}

\subsection{Evolution and structure}

\subsubsection{Evolution}

\subsubsection{Disk structure}

\subsubsection{Multi-component jets}

\subsection{Models and simulations}

\subsubsection{Disk winds}

\subsubsection{X-Wind}

\subsubsection{Stellar wind and magnetic braking}
Stellar winds got out of fashion because the thermal driving is not sufficient to explain the high mass loss rates and velocities observed. However, a complete description of the jet cannot exclude this inner part and moreover the stellar outflow is nowadays the only issue to the angular momentum problem.
\fxnote{Figure}
T-Tauri stars are rotation at 10\% of their breakup speed. Classical T-Tauri stars are rotating more slowly than weak T-Tauri stars. It was originally thought that the difference could come from the connection between the CTTS and its disk, which is absent for WTTS.

This is the so-called disk locking. A magnetic line of the star rooted in the disk beyond the co-rotation radius spins down the rotation of the star, while a line rooted below the co-rotation radius will spin it up. However simulations shown that the disk locking is inefficient because most of the lines are below the co-rotation radius in a stable configuration.

Magnetic braking (see \ref{fix ref} next section) from the magnetized stellar wind is more efficient. The following figures \fxnote{put them} illustrate a stellar jet solution collimating on large distances, adapted for low mass accretion rates as in RY Tau. The mass loss rate from the stellar wind is $\SI{e-9}{M_\odot\per\year}$. Collimation is magnetic but the outflow is pressure driven.

THe density profile is compatible with observations in the jet of RY Tau. The temperature calculated as the ratio of the pressure to density is high. However, if the wind is pressure driven, it does mean that the total pressure is the kinetic component only. The pressure is the sum of various contributions (ram pressure, Alfvén waves, etc…). This the plotted temperature is an effective temperature larger than the real temperature.

\fxnote{Figures}

These analytical solution can be used as initial condition. The magnetosphere can be treated as a dead zone.

\fxnote{Figures}

The end result is to obtain sporadic X-wind but with steady accretion and stellar jet conversely to previously.

\paragraph{Braking time}

We can easily estimate the braking time of the central object (Schatzmann 1957, Weber \& Davis 1967, Mestel 1968). The angular momentum loss extracted by the wind is:
\begin{equation}
  \dot{J}_\star = \frac{2}{3}\varpi_A^2\Omega \dot{M}_\mathrm{wind}
\end{equation}
The total stellar angular momentum is
\begin{equation}
  J_\star = kr^2_\star \Omega M_\star
\end{equation}
$k$ depends on the stellar structure, \num{0.06} for the sun up to \num{0.2} for a fully convective star. This to remove the angular momentum of the star, it takes:
\begin{equation}
  \tau \approx \frac{J_\star}{\dot{J}_\star} \approx \frac{3kr^2_\star \Omega M_\star}{2\varpi_A^2\Omega\dot{M}_\mathrm{wind}} \approx \frac{3 kr_\star^2 M_\star}{2\varpi_A^2\dot{M}_\mathrm{wind}}
\end{equation}

\section{Counter rotation}
\label{sec:counter_rotation}



\section{Extra-Galactic Nuclei}
\section{X-ray binaries and microquasars}

\appendix

\makeatletter
\def\@seccntformat#1{Appendix~\csname the#1\endcsname:\quad}
\makeatother

\newpage

\bibliographystyle{plainnat}
\bibliography{ET8}
\addcontentsline{toc}{section}{References}

\listoffixmes

\end{document}
